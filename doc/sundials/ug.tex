\usepackage{makeidx}
\usepackage{xtab}
\usepackage{amsmath, amssymb}

% Package to add Bibliography and Index to TOC
% but not the TOC itself :-)
\usepackage[nottoc]{tocbibind}

\usepackage{calc}

% Packages for adding figures
\usepackage{subfigure}
\usepackage{color}
\usepackage{graphicx}

% Needed for sideways
\usepackage{rotating}

% Packages used for output and source code listings
\usepackage{fancyvrb}
\usepackage{listings}

%\usepackage{fancyheadings}
\usepackage{fancyhdr}

% External references
\usepackage{xr}

% show margin lines for debugging
%\usepackage[showframe]{geometry}

%----- Define some colors
\definecolor{gray}{rgb}{0.5,0.5,0.5}

%----- Define the warning sign

\newcommand{\marginlabel}[1]
{\mbox{}\marginpar{\raggedleft\hspace{0pt}{\rule{0mm}{1mm}#1}}}

\newcommand{\warn}{%
%\marginlabel{\pstribox[shadow=false,fillstyle=solid,fillcolor=yellow,trimode=*U]{\footnotesize !}}%%
\marginlabel{\includegraphics[scale=0.8]{warning}}%%
}

%----- Page formatting

\setlength{\paperwidth}{8.5in}
\setlength{\textwidth}{6.1in}
\setlength{\hoffset}{0.0in}
\setlength{\oddsidemargin}{0.2in}
\setlength{\evensidemargin}{0.2in}
\setlength{\marginparsep}{0.25in}
\setlength{\marginparwidth}{0.5in}

\setlength{\paperheight}{11.0in}
\setlength{\textheight}{9.0in}
\setlength{\voffset}{-0.25in}
\setlength{\topmargin}{0.0in}

%----- VERSIONS AND UCRL NUMBERS OF SUNDIALS CODES

\newcommand{\sunrelease}{v6.4.1}

\newcommand{\cvrelease}{v6.4.1}
\newcommand{\cvucrlug}{UCRL-SM-208108}
\newcommand{\cvucrlex}{UCRL-SM-208110}

\newcommand{\cvsrelease}{v6.4.1}
\newcommand{\cvsucrlug}{UCRL-SM-208111}
\newcommand{\cvsucrlex}{UCRL-SM-208115}

\newcommand{\idarelease}{v6.4.1}
\newcommand{\idaucrlug}{UCRL-SM-208112}
\newcommand{\idaucrlex}{UCRL-SM-208113}

\newcommand{\idasrelease}{v5.4.1}
\newcommand{\idasucrlug}{UCRL-SM-234051}
\newcommand{\idasucrlex}{LLNL-TR-437091}

\newcommand{\kinrelease}{v6.4.1}
\newcommand{\kinucrlug}{UCRL-SM-208116}
\newcommand{\kinucrlex}{UCRL-SM-208114}

\newcommand{\arkrelease}{v5.4.1}
\newcommand{\arkucrlug}{LLNL-SM-668082}
\newcommand{\arkucrlex}{????-??-??????}

%----- SUNDIALS MODULES
\newcommand{\sundials}{{\normalfont\scshape sundials}}
\newcommand{\shared}{{\normalfont\scshape shared}}
%----- Vectors:
\newcommand{\nvector}{{\normalfont\scshape nvector}}
\newcommand{\fnvector}{{\normalfont\scshape fnvector}}
\newcommand{\nvecs}{{\normalfont\scshape nvector\_serial}}
\newcommand{\nvecp}{{\normalfont\scshape nvector\_parallel}}
\newcommand{\nvecopenmp}{{\normalfont\scshape nvector\_openmp}}
\newcommand{\nvecopenmpdev}{{\normalfont\scshape nvector\_openmpdev}}
\newcommand{\nvecpthreads}{{\normalfont\scshape nvector\_pthreads}}
\newcommand{\nvecph}{{\normalfont\scshape nvector\_parhyp}}
\newcommand{\nvecpetsc}{{\normalfont\scshape nvector\_petsc}}
\newcommand{\nveccuda}{{\normalfont\scshape nvector\_cuda}}
\newcommand{\nvechip}{{\normalfont\scshape nvector\_hip}}
\newcommand{\nvecraja}{{\normalfont\scshape nvector\_raja}}
\newcommand{\nvecsycl}{{\normalfont\scshape nvector\_sycl}}
\newcommand{\nvectrilinos}{{\normalfont\scshape nvector\_trilinos}}
\newcommand{\nvecwrap}{{\normalfont\scshape nvector\_senswrapper}}
\newcommand{\nvecmanyvector}{{\normalfont\scshape nvector\_manyvector}}
\newcommand{\nvecmpimanyvector}{{\normalfont\scshape nvector\_mpimanyvector}}
\newcommand{\nvecmpiplusx}{{\normalfont\scshape nvector\_mpiplusx}}
%----- Matrices:
\newcommand{\sunmatrix}{{\normalfont\scshape sunmatrix}}
\newcommand{\fsunmatrix}{{\normalfont\scshape fsunmatrix}}
\newcommand{\sunmatband}{{\normalfont\scshape sunmatrix\_band}}
\newcommand{\sunmatdense}{{\normalfont\scshape sunmatrix\_dense}}
\newcommand{\sunmatsparse}{{\normalfont\scshape sunmatrix\_sparse}}
\newcommand{\sunmatcusparse}{{\normalfont\scshape sunmatrix\_cusparse}}
\newcommand{\sunmatslunrloc}{{\normalfont\scshape sunmatrix\_slunrloc}}
%----- Linear Solvers:
\newcommand{\sunlinsol}{{\normalfont\scshape sunlinsol}}
\newcommand{\fsunlinsol}{{\normalfont\scshape fsunlinsol}}
\newcommand{\sunlinsolband}{{\normalfont\scshape sunlinsol\_band}}
\newcommand{\sunlinsoldense}{{\normalfont\scshape sunlinsol\_dense}}
\newcommand{\sunlinsollapband}{{\normalfont\scshape sunlinsol\_lapackband}}
\newcommand{\sunlinsollapdense}{{\normalfont\scshape sunlinsol\_lapackdense}}
\newcommand{\sunlinsolklu}{{\normalfont\scshape sunlinsol\_klu}}
\newcommand{\sunlinsolsludist}{{\normalfont\scshape sunlinsol\_superludist}}
\newcommand{\sunlinsolslumt}{{\normalfont\scshape sunlinsol\_superlumt}}
\newcommand{\sunlinsolcuspbqr}{{\normalfont\scshape sunlinsol\_cusolversp\_batchqr}}
\newcommand{\sunlinsolspgmr}{{\normalfont\scshape sunlinsol\_spgmr}}
\newcommand{\sunlinsolspfgmr}{{\normalfont\scshape sunlinsol\_spfgmr}}
\newcommand{\sunlinsolspbcgs}{{\normalfont\scshape sunlinsol\_spbcgs}}
\newcommand{\sunlinsolsptfqmr}{{\normalfont\scshape sunlinsol\_sptfqmr}}
\newcommand{\sunlinsolpcg}{{\normalfont\scshape sunlinsol\_pcg}}
\newcommand{\sunlinsolcusparse}{{\normalfont\scshape sunlinsol\_cusparse}}
%----- Nonlinear Solvers:
\newcommand{\sunnonlinsol}{{\normalfont\scshape sunnonlinsol}}
\newcommand{\fsunnonlinsol}{{\normalfont\scshape fsunnonlinsol}}
\newcommand{\sunnonlinsolnewton}{{\normalfont\scshape sunnonlinsol\_newton}}
\newcommand{\sunnonlinsolfixedpoint}{{\normalfont\scshape sunnonlinsol\_fixedpoint}}
\newcommand{\sunnonlinsolpetsc}{{\normalfont\scshape sunnonlinsol\_petscsnes}}
%----- SUNMemory:
\newcommand{\sunmemory}{{\normalfont\scshape sunmemory}}
%----- Packages:
\newcommand{\cvode}{{\normalfont\scshape cvode}}
\newcommand{\pvode}{{\normalfont\scshape pvode}}
\newcommand{\cvodes}{{\normalfont\scshape cvodes}}
\newcommand{\arkode}{{\normalfont\scshape arkode}}
\newcommand{\ida}{{\normalfont\scshape ida}}
\newcommand{\idas}{{\normalfont\scshape idas}}
\newcommand{\kinsol}{{\normalfont\scshape kinsol}}
\newcommand{\sundialsTB}{{\normalfont\scshape sundialsTB}}

%----- OTHER PACKAGES
\newcommand{\vode}{{\normalfont\scshape vode}}
\newcommand{\vodpk}{{\normalfont\scshape vodpk}}
\newcommand{\lsode}{{\normalfont\scshape lsode}}
\newcommand{\daspk}{{\normalfont\scshape daspk}}
\newcommand{\daspkadjoint}{{\normalfont\scshape daspkadjoint}}
\newcommand{\hypre}{\textsl{hypre}}
\newcommand{\openmp}{{\normalfont OpenMP}}
\newcommand{\petsc}{{\normalfont\scshape pets}c}
\newcommand{\cuda}{{\normalfont\scshape cuda}}
\newcommand{\hip}{{\normalfont\scshape hip}}
\newcommand{\raja}{{\normalfont\scshape raja}}
\newcommand{\sycl}{{\normalfont\scshape sycl}}
\newcommand{\superludist}{SuperLU\_DIST}
\newcommand{\trilinos}{Trilinos}
\newcommand{\tpetra}{Tpetra}

%----- CVODE and CVODES COMPONENTS
\newcommand{\cvnls}{{\normalfont\scshape cvnls}}
\newcommand{\cvdls}{{\normalfont\scshape cvdls}}
\newcommand{\cvsls}{{\normalfont\scshape cvsls}}
\newcommand{\cvcls}{{\normalfont\scshape cvcls}}
\newcommand{\cvls}{{\normalfont\scshape cvls}}
\newcommand{\cvdense}{{\normalfont\scshape cvdense}}
\newcommand{\cvsparse}{{\normalfont\scshape cvsparse}}
\newcommand{\cvband}{{\normalfont\scshape cvband}}
\newcommand{\cvdiag}{{\normalfont\scshape cvdiag}}
\newcommand{\cvspils}{{\normalfont\scshape cvspils}}
\newcommand{\cvspgmr}{{\normalfont\scshape cvspgmr}}
\newcommand{\cvspbcg}{{\normalfont\scshape cvspbcg}}
\newcommand{\cvsptfqmr}{{\normalfont\scshape cvsptfqmr}}
\newcommand{\cvklu}{{\normalfont\scshape cvklu}}
\newcommand{\cvsuperlumt}{{\normalfont\scshape cvsuperlumt}}
\newcommand{\cvbandpre}{{\normalfont\scshape cvbandpre}}
\newcommand{\cvbbdpre}{{\normalfont\scshape cvbbdpre}}
\newcommand{\cvodea}{{\normalfont\scshape cvodea}}
\newcommand{\fcvode}{{\normalfont\scshape fcvode}}
\newcommand{\fcvbp}{{\normalfont\scshape fcvbp}}
\newcommand{\fcvbbd}{{\normalfont\scshape fcvbbd}}
\newcommand{\fcvroot}{{\normalfont\scshape fcvroot}}
\newcommand{\stald}{{\normalfont\scshape stald}}

%----- KINSOL COMPONENTS
\newcommand{\kinspils}{{\normalfont\scshape kinspils}}
\newcommand{\kinspgmr}{{\normalfont\scshape kinspgmr}}
\newcommand{\kinspfgmr}{{\normalfont\scshape kinspfgmr}}
\newcommand{\kinspbcg}{{\normalfont\scshape kinspbcg}}
\newcommand{\kinsptfqmr}{{\normalfont\scshape kinsptfqmr}}
\newcommand{\kinbbdpre}{{\normalfont\scshape kinbbdpre}}
\newcommand{\kindls}{{\normalfont\scshape kindls}}
\newcommand{\kincls}{{\normalfont\scshape kincls}}
\newcommand{\kinsls}{{\normalfont\scshape kinsls}}
\newcommand{\kinls}{{\normalfont\scshape kinls}}
\newcommand{\kindense}{{\normalfont\scshape kindense}}
\newcommand{\kinband}{{\normalfont\scshape kinband}}
\newcommand{\kinsparse}{{\normalfont\scshape kinsparse}}
\newcommand{\kinklu}{{\normalfont\scshape kinklu}}
\newcommand{\kinsuperlumt}{{\normalfont\scshape kinsuperlumt}}
\newcommand{\fkinbbd}{{\normalfont\scshape fkinbbd}}
\newcommand{\fkindense}{{\normalfont\scshape fkindense}}
\newcommand{\fkinband}{{\normalfont\scshape fkinband}}
\newcommand{\fkinsol}{{\normalfont\scshape fkinsol}}

%----- IDA and IDAS COMPONENTS
\newcommand{\idanls}{{\normalfont\scshape idanls}}
\newcommand{\idadls}{{\normalfont\scshape idadls}}
\newcommand{\idacls}{{\normalfont\scshape idacls}}
\newcommand{\idasls}{{\normalfont\scshape idasls}}
\newcommand{\idals}{{\normalfont\scshape idals}}
\newcommand{\idadense}{{\normalfont\scshape idadense}}
\newcommand{\idasparse}{{\normalfont\scshape idasparse}}
\newcommand{\idaband}{{\normalfont\scshape idaband}}
\newcommand{\idaklu}{{\normalfont\scshape idaklu}}
\newcommand{\idasuperlumt}{{\normalfont\scshape idasuperlumt}}
\newcommand{\idaspils}{{\normalfont\scshape idaspils}}
\newcommand{\idaspgmr}{{\normalfont\scshape idaspgmr}}
\newcommand{\idaspbcg}{{\normalfont\scshape idaspbcg}}
\newcommand{\idasptfqmr}{{\normalfont\scshape idasptfqmr}}
\newcommand{\idabbdpre}{{\normalfont\scshape idabbdpre}}
\newcommand{\idaa}{{\normalfont\scshape idaa}}
\newcommand{\fida}{{\normalfont\scshape fida}}
\newcommand{\fidaband}{{\normalfont\scshape fidaband}}
\newcommand{\fidadense}{{\normalfont\scshape fidadense}}
\newcommand{\fidabbd}{{\normalfont\scshape fidabbd}}
\newcommand{\fidaroot}{{\normalfont\scshape fidaroot}}

%----- SHARED COMPONENTS
\newcommand{\sundialsmath}{{\normalfont\scshape sundialsmath}}
\newcommand{\diag}{{\normalfont\scshape diag}}
\newcommand{\dense}{{\normalfont\scshape dense}}
\newcommand{\band}{{\normalfont\scshape band}}
\newcommand{\pcg}{{\normalfont\scshape pcg}}
\newcommand{\sls}{{\normalfont\scshape sls}}
\newcommand{\sparse}{{\normalfont\scshape sparse}}
\newcommand{\spgmr}{{\normalfont\scshape spgmr}}
\newcommand{\spfgmr}{{\normalfont\scshape spfgmr}}
\newcommand{\spbcg}{{\normalfont\scshape spbcgs}}
\newcommand{\spbcgs}{{\normalfont\scshape spbcgs}}
\newcommand{\sptfqmr}{{\normalfont\scshape sptfqmr}}
\newcommand{\klu}{{\normalfont\scshape klu}}
\newcommand{\superlumt}{{\normalfont\scshape superlumt}}

%----- OS
\newcommand{\linux}{{\sc LINUX}}
\newcommand{\unix}{{\sc UNIX}}
\newcommand{\mpi}{{\sc MPI}}

%----- C and Fortran languages
\newcommand{\CC}{{\sc C}}
\newcommand{\CPP}{{\sc C}\raisebox{0.2em}{\tiny ++}}
\newcommand{\F}{{\sc Fortran}}

%------ Serial, OpenMP, Pthreads, Parallel, or ParHyp
\newcommand{\s}{[{\bf S}]}
\newcommand{\omp}{[{\bf O}]}
\newcommand{\pt}{[{\bf T}]}
\newcommand{\p}{[{\bf P}]}
\newcommand{\h}{[{\bf H}]}

%------ Appendix in text
\newcommand{\A}{App. }

%------ Index entries

\newcommand{\ID}[1]{{\tt #1}{\index{#1@\texttt{#1}}}}
%\newcommand{\ID}[1]{{\tt #1}{\index{#1@\texttt{#1}|textbf}}}
% NOTE: the above version confuses hyperref...
\newcommand{\Id}[1]{{\tt #1}{\index{#1@\texttt{#1}}}}
\newcommand{\id}[1]{{\tt #1}}

%%----- Shortcuts for math formulas
\newcommand{\mb}[1]{{\mbox{\scriptsize #1}}}

\newcommand{\dfdy}{\frac{\partial f}{\partial y}}
\newcommand{\dfdyI}{\partial f / \partial y}
\newcommand{\dfdpi}{\frac{\partial f}{\partial p_i}}
\newcommand{\dfdpiI}{\partial f / \partial p_i}

\newcommand{\dFdy}{\frac{\partial F}{\partial y}}
\newcommand{\dFdyI}{\partial F / \partial y}
\newcommand{\dFdyp}{\frac{\partial F}{\partial \dot y}}
\newcommand{\dFdypI}{\partial F / \partial \dot y}
\newcommand{\dFdpi}{\frac{\partial F}{\partial p_i}}
\newcommand{\dFdpiI}{\partial F / \partial p_i}

\newcommand{\rhomax}{\rho_{\max}}

\newcommand{\cm}{$\checkmark$}

\newcommand{\newlinefill}[1]{\newline\phantom{#1}}

%%------ Specify depths for section numbering and TOC

\setcounter{tocdepth}{1}
\setcounter{secnumdepth}{3}

%%-------

\newcommand{\includeCode}[1]
{
  \VerbatimInput[numbers=left,fontsize=\small]{#1}
}

\newcommand{\includeOutput}[2]
{
  \vspace{0.1in}
  \VerbatimInput[frame=single,
                 framesep=0.1in,
                 label={\tt #1} sample output,
                 fontsize=\footnotesize]{#2}
}

%%-------

\newcommand{\ugref}[1]{\S\ref{#1}}

%%-------

\newcommand{\clearemptydoublepage}{\newpage{\pagestyle{empty}\cleardoublepage}}

%%-------

\newcommand{\disclaimer}{%
\thispagestyle{empty}% no number of this page
\vglue5\baselineskip
\begin{center}
{\bf DISCLAIMER}
\end{center}
\noindent
This document was prepared as an account of work sponsored by an agency of
the United States government. Neither the United States government nor
Lawrence Livermore National Security, LLC, nor any of their employees makes
any warranty, expressed or implied, or assumes any legal liability or responsibility
for the accuracy, completeness, or usefulness of any information, apparatus, product,
or process disclosed, or represents that its use would not infringe privately owned rights.
Reference herein to any specific commercial product, process, or service by trade name,
trademark, manufacturer, or otherwise does not necessarily constitute or imply its endorsement,
recommendation, or favoring by the United States government or Lawrence Livermore National
Security, LLC. The views and opinions of authors expressed herein do not necessarily state
or reflect those of the United States government or Lawrence Livermore National Security, LLC,
and shall not be used for advertising or product endorsement purposes.

\vskip2\baselineskip
\noindent This work was performed under the auspices of the U.S. Department of Energy by Lawrence
Livermore National Laboratory under Contract DE-AC52-07NA27344.
\vfill
\begin{center}
Approved for public release; further dissemination unlimited
\end{center}
\clearpage
}

%%-------

\newcommand{\contributors}{%
\thispagestyle{empty}% no number of this page
\vglue5\baselineskip
\begin{center}
{\bf CONTRIBUTORS}
\end{center}
\noindent
The SUNDIALS library has been developed over many years by a number of
contributors. The current SUNDIALS team consists of Cody J. Balos,
David J. Gardner, Alan C. Hindmarsh, Daniel R. Reynolds, and
Carol S. Woodward. We thank Radu Serban for significant and critical past
contributions.\\
\\
\noindent
Other contributors to SUNDIALS include: James Almgren-Bell, Lawrence E. Banks,
Peter N. Brown, George Byrne, Rujeko Chinomona, Scott D. Cohen, Aaron Collier,
Keith E. Grant, Steven L. Lee, Shelby L. Lockhart, John Loffeld, Daniel McGreer,
Slaven Peles, Cosmin Petra, H. Hunter Schwartz, Jean M. Sexton,
Dan Shumaker, Steve G. Smith, Allan G. Taylor, Hilari C. Tiedeman, Chris White,
Ting Yan, and Ulrike M. Yang.
\clearpage
}

%%--------------------------------

\newcommand{\frontug}
{

  %% Start roman numbering
  \pagenumbering{roman}

  %% Title page
  \maketitle
  \disclaimer
  \contributors
  \clearemptydoublepage

  %% Contents, tables, and figures
  \tableofcontents
  \clearemptydoublepage
  \listoftables
  \clearemptydoublepage
  \listoffigures
  \clearemptydoublepage

  %% Start arabic numbering
  \pagenumbering{arabic}

  %% Define page style for the body of the document
  \lhead[\fancyplain{}{\bfseries\thepage}]%
        {\fancyplain{}{\bfseries\rightmark}}
  \rhead[\fancyplain{}{\bfseries\leftmark}]%
        {\fancyplain{}{\bfseries\thepage}}
  \cfoot{}
  \pagestyle{fancyplain}


  %% Some settings for code listings
  \lstset{
    language=C, fancyvrb=true,
    basicstyle=\footnotesize\ttfamily,
    commentstyle=\color{MidnightBlue},
    keywordstyle=\color{BrickRed},
    stringstyle=\color{OliveGreen},
    numbers=left, numberstyle=\tiny, numbersep=15pt}

}

%%--------------------------------
\newcommand{\frontex}
{

  \pagestyle{empty}
  \maketitle
  \disclaimer
  \contributors
  \clearemptydoublepage

  % Contents
  \tableofcontents

  %% Start arabic numbering
  \clearemptydoublepage
  %%\clearpage
  \pagestyle{plain}\pagenumbering{arabic}

  %% Some settings for code listings
  \lstset{
    fancyvrb=true,
    basicstyle=\footnotesize\ttfamily,
    commentstyle=\color{MidnightBlue},
    keywordstyle=\color{BrickRed},
    stringstyle=\color{OliveGreen},
    numbers=left, numberstyle=\tiny, numbersep=15pt}

}

%%----------------------------------
%% List of configure options
%%---------------------------------
\newenvironment{config}
{\begin{list}{}{
      \setlength{\leftmargin}{2em}
      \setlength{\rightmargin}{0em}
      \setlength{\topsep}{0.05in}
      \setlength{\itemindent}{-2em}
      \setlength{\itemsep}{0.05in}}}
  {\end{list}}
%%---------------------------------
%% Steps used in skeleton programs
%%---------------------------------
\newcounter{Stepsctr}
\newenvironment{Steps}
{\stepcounter{Stepsctr}
  \begin{list}{\arabic{Stepsctr}. }{
      \usecounter{Stepsctr}
      \setlength{\parsep}{0.5em}
      \setlength{\labelsep}{0em}
      \settowidth{\labelwidth}{99. }
      \setlength{\leftmargin}{\labelwidth+\labelsep}}}
  {\end{list}}
%%-----------------------------------------------------
%% Underlying list environemnt for function definitions
%%-----------------------------------------------------
\newenvironment{Ventry}[1][\quad]
{\begin{list}{}{
      \setlength{\rightmargin}{0em}
      \setlength{\topsep}{0.05in}
      \setlength{\itemsep}{0em}
      \setlength{\itemindent}{0em}
      \setlength{\labelsep}{0.5em}
      \renewcommand{\makelabel}[1]{##1\hfill}
      \settowidth{\labelwidth}{#1}
      \setlength{\leftmargin}{\labelwidth+\labelsep}}}
  {\end{list}}
%%----------------------------------
%% List of function arguments
%%---------------------------------
\newenvironment{args}[1][\quad]
{\begin{list}{}{
      \setlength{\rightmargin}{0em}
      \setlength{\topsep}{0em}
      \setlength{\itemsep}{0em}
      \setlength{\itemindent}{0em}
      \setlength{\labelsep}{0.5em}
      \renewcommand{\makelabel}[1]{\id{##1}\hfill}
      \settowidth{\labelwidth}{\id{#1}}
      \setlength{\leftmargin}{\labelwidth+\labelsep}}}
  {\end{list}}
%%---------------------------------
%% User-callable function
%%---------------------------------
\newcommand{\ucfunction}[6]{
  \noindent\paragraph{\fbox{\id{#1}}}
  \begin{Ventry}[Return value]
  \item[Call]{\id{#2}}
  \item[Description]{#3}
  \addArgs{#4}
  \addRetval{#5}
  \addNotes{#6}
  \end{Ventry}
}
%%----------------------------------------------
%% User-callable function with a deprecated name
%%----------------------------------------------
\newcommand{\ucfunctiond}[7]{
  \noindent\paragraph{\fbox{\id{#1}}}
  \begin{Ventry}[Deprecated Name]
  \item[Call]{\id{#2}}
  \item[Description]{#3}
  \addArgs{#4}
  \addRetval{#5}
  \addNotes{#6}
  \addDepName{#7}
  \end{Ventry}
}
%%-----------------------------------------------------
%% User-callable function with a Fortran 2003 interface
%%-----------------------------------------------------
\newcommand{\ucfunctionf}[6]{
  \noindent\paragraph{\fbox{\id{#1}}}
  \begin{Ventry}[Return value]
  \item[Call]{\id{\begin{tabular}[t]{@{}r@{}l@{}}#2\end{tabular}}}
  \item[Description]{#3}
  \addArgs{#4}
  \addRetval{#5}
  \addNotes{#6}
  \item[F2003 Name]{\id{F#1}}
  \end{Ventry}
}
%%-----------------------------------------------------
%% User-callable function with a Fortran 2003 interface
%% and an additional entry on how to call the F2003
%% function.
%%-----------------------------------------------------
\newcommand{\ucfunctionfl}[7]{
  \noindent\paragraph{\fbox{\id{#1}}}
  \begin{Ventry}[Return value]
  \item[Call]{\id{#2}}
  \item[Description]{#3}
  \addArgs{#4}
  \addRetval{#5}
  \addNotes{#6}
  \item[F2003 Name]{\id{F#1}}
  \item[F2003 Call]{\id{#7}}
  \end{Ventry}
}
%%-----------------------------------------------------
%% User-callable function with a deprecated name and
%% a Fortran 2003 interface
%%-----------------------------------------------------
\newcommand{\ucfunctiondf}[7]{
  \noindent\paragraph{\fbox{\id{#1}}}
  \begin{Ventry}[Deprecated Name]
  \item[Call]{\id{#2}}
  \item[Description]{#3}
  \addArgs{#4}
  \addRetval{#5}
  \addNotes{#6}
  \addDepName{#7}
  \item[F2003 Name]{\id{F#1}}
  \end{Ventry}
}
%%--------------------------------------------
%% macro for generic SUNDIALS module functions
%%--------------------------------------------
\newcommand{\sunmodfun}[3]{
  \noindent\paragraph{\fbox{\ID{#1}}}
  \begin{Ventry}[F2003 Name]
  \item[Prototype]{\id{#3}}
  \item[Description]{#2}
  \end{Ventry}
}
%%-------------------------------------------------------
%% macro for generic overloaded SUNDIALS module functions
%%-------------------------------------------------------
\newcommand{\sunmodfuns}[6]{
  \noindent\paragraph{\fbox{\ID{#1}}}
  \begin{Ventry}[F2003 Name]
  \item[\textit{Single-node usage}]{}
  \item[Prototype]{\id{#4}}
  \item[Description]{#2#3}
  \item[\textit{Distributed-memory parallel usage}]{}
  \item[Prototype]{\id{#6}}
  \item[Description]{#2#5}
  \end{Ventry}
}
%%--------------------------------------------------------------------------
%% macro for generic SUNDIALS module functions with a Fortran 2003 interface
%%--------------------------------------------------------------------------
\newcommand{\sunmodfunf}[3]{
  \noindent\paragraph{\fbox{\ID{#1}}}
  \begin{Ventry}[F2003 Name]
  \item[Prototype]{\id{#3}}
  \item[Description]{#2}
  \item[F2003 Name]{This function is callable as \id{F#1} when using the Fortran
  2003 interface module.}
  \end{Ventry}
}
%%---------------------------------
%% User-supplied function
%%---------------------------------
\newcommand{\usfunction}[6]{
  \noindent\paragraph{\fbox{\id{#1}}}
  \begin{Ventry}[Return value]
  \item[Definition]{\id{\begin{tabular}[t]{@{}r@{}l@{}}#2\end{tabular}}}
  \item[Purpose]{#3}
  \item[Arguments]{#4}
  \item[Return value]{#5}
  \addNotes{#6}
  \end{Ventry}
}
%%---------------------------------
\makeatletter
\long\def\addArgs#1{\def\@tempa{#1}\ifx\@tempa\empty\item[Arguments]{None}\else\item[Arguments]{#1}\fi}
\makeatother
%%---------------------------------
\makeatletter
\long\def\addRetval#1{\def\@tempa{#1}\ifx\@tempa\empty\item[Return value]{None}\else\item[Return value]{#1}\fi}
\makeatother
%%---------------------------------
\makeatletter
\long\def\addNotes#1{\def\@tempa{#1}\ifx\@tempa\empty\else\item[Notes]{#1}\fi}
\makeatother
%%---------------------------------
\makeatletter
\long\def\addDepName#1{\def\@tempa{#1}\ifx\@tempa\empty\item[Deprecated
  Name]{None}\else\item[Deprecated Name]{For backward compatibility, the wrapper
  function \id{#1} with idential input and output arguments is also
  provided.}\fi}
\makeatother


%%---------------------------------
%% Finally, use hyperref package to include links in the PDF
%% Note that since hyperref redefines a bunch of LaTeX commands,
%% to give it a fighting chance, it MUST be included last.

\usepackage[
% letterpaper=true,
% dvips,
% ps2pdf,
hyperindex=true,
linktocpage=true,
colorlinks=true,
linkcolor=blue,
citecolor=blue,
bookmarks=true]
{hyperref}
